\section{Problemstellung}
Die beschränkenden Faktoren für das Wachstum des Internet stellen die Größe und die Anzahl der Updates den Corerouting-Tabelle. Letzteres impliziert vor allem mit wie viel Aufwand die Corerouter ihre eigentlich Arbeit, das Weiterleiten von Paketen durch führen können \cite{Huston:2003:BGP}.

% Kernproblem:
\paragraph{}
Das Kernproblem des Internet-Routings ist das die Anzahl der Edge-Networks die Größe der Routing-Tablle bestimmt. Eine Zunahme um ca. 2500 regestrierte Addressprefixe pro Jahr \cite{journals/ccr/MengXZHLZ04} und die anstehende Vergrößerung des Addressraums durch die Einführung von IPv6 gefärhden die Betriebsfähigkeit des bestehnden Routing-Systems \cite{jen:2008:start}.

%Gründe des Kernproblems
\paragraph{}
Die hohe Anzahl von Edge-Network-Einträgen in den Corerouting-Tabllen rührt daher, dass Adressprefixe nicht hierachisch vergeben werden. Bei hirachischer Vergabe erhalten die Edgenetworks einen Prefix, welcher innherlb des Prefix des Providers liegt. Dadurch muss Prefix des ENs nicht in die Corerouting-Tabelle eingetragen werden. \\
Die Hierachische Vergabe birgt allerdings Nachteile aus Sicht der Edgenetworks. So wird zum Beispiel der Wechsel des ISP erschwert, da in diesem Fall das EN einen anderen Prefix erhalten würde. Um dies zu verhindern benutzen die Edgenetworks eine providerunabhänige Addresse (PI). Um unter dieser erreichbar zu sein ist es jedoch nötig den gesamten Internet-Core über eine Route zum Prefix des ENs zu informieren. \\
Ein weiters Problem bei der hirachische Vergabe sind Verbindungen eines ENs zu mehren Providern (Multihoming). Multihoming ermöglich Lastausgleich zwischen Verbindungen oder Redundanz falls eine Verbindung ausfällt. Allerdings müssen beide Routen dem gesamten Internet-Core bekannst sein. Daher benötigt ein EN für Multihoming eine providerunabhänige Adresse. Im Jahr 1998 enthielten die BGP-Tabllen pro regstrietem Prefix 1,33 Einträge. 2004 betrug diese Zahl 2,54 Einträge pro Netblock, dies zeigt das die Anzahl der Netze mit Multihoming zunimmt.

\paragraph{Größe der Routing-Tabellen}
Im Jahr 2004 wurden seit den Anfangsjahren des Internets ca. 63 000 IP-Blöcke bei registriert, davon ungefähr 18 000 in den letzten 7 Jahren \cite{journals/ccr/MengXZHLZ04}. Gleichzeitig enthielt die BGP Routing Tablle über 160 000 Einträge, was einer Anzahl von 2,54 Einträgen pro regeneriertem Adressblock entspricht, im Vergleich dazu waren es 1998 noch 1,33 Einträge pro Adressblock. Die hohe Dichte an Routen deutet auf eine hohe Anzahl von edge networks hin, die über mehrere Routen zu erreichen. Solche EN bezeichnet man als multihomed. Sie sind über mehrere ISPs verbunden um die Netzwerklast auf diese zu verteilen oder aber auch um eine Redundanz der Verbindung zu erhalten. Damit dies möglich ist muss der IP-Block des jeweiligen ENs in der globalen Routing-Tabelle sichtbar sein, was zur Folge hat das kein Provider dieses edge network in seinen eignenen Block auf nehmen kann \cite{jen:2008:start}. Das EN benötigt eine Provider unabhängige Adresse (engl. provider independet address, PI). Wäre der Adressblock nicht für alle Router sichtbar, so müsste das EN seine Geräte mit Adressen von allen Providern die es benutzt versehen. Dies würde dazu führen das im Falle eines Providerwechsels eine Neunummerierung des gesamten Netzwerkes statt finden müsste. Viel gravierende ist jedoch das keine echte Redundanz vorhanden wäre, da wenn eine Verbindung zu einem Provider ausfällt das Netzwerk nur erreicht werden kann, wenn die Gegenstelle auch die anderen Adressen des Netzwerkes kennt.

\paragraph{Änderung der Routing-Informationen} 
Neben der hohen Anzahl an Einträgen in den Routing-Tabellen selbst, gefährdet auch die Anzahl der Änderungen die Skalierbarkeit des Systems Beispielsweise wurden in der ersten Jahreshälfte 2004 24000 Einträge in der BGP-Tabellen entfernt, im gleichen Zeitraum wurde jedoch auch 36000 Einträge hinzugefügt, die Anzahl der erreichbaren Geräte veränderte sich aber nur geringfügig. Dabei kommt ein hoher Anteil der Änderung von nur wenigen AS, so erreichten vom 14. Dezember bis zum 20. Dezember 2009 insgesamt 465992 BGP updates die BGP Tabellen, wovon 30,33 \% von nur 50 verschiedenen AS getätigt wurden \cite{Huston:aktuell:BGP}. Änderungen der BGP Informationen werden vor allem durch Providerwechsel von edge networks und von fehl konfigurierte BGP-Router provoziert. Die Einträge der größeren ISPs unterliegen vergleichsweise wenigen Änderungen. Hinzu kommt das Änderungen erst dann für AS interessant sind, wenn Verkehr in den betroffenen Netzwerkblock geleitet werden soll.



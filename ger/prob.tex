\section{Problemstellung}


%  kurze Begriffsklärung Einteilung
\paragraph{}
Das Internet besteht aus kleineren individuellen selbständigen Einheiten. Ein solches Netzwerk wird als AS (Autonomes System) bezeichnet, und unterliegt jeweils einer Institution. Ein Autonomes System welches Daten von anderen ASs weiterleitet wird als ISP (Internet Service Provider) bezeichnet, andernfalls handelt es sich um ein EN  (Edgenetwork) \cite{Mahorta:2002:IR}. Der Internet-Core (oder auch Internet-Backbone) wird durch alle ISPs gebildet.\\

% Routing
Mit Routing wir der Vorgang bezeichnet der nötig ist um in einem Netzwerk ein Datenpaket von einem Router zum nächsten weiterzuleiten \cite{Mahorta:2002:IR}. Auf dem Weg, den das Paket durch ein Netzwerk nimmt, muss an jedem Router entschieden werden, zu welchem seiner direkten Nachbarn (nächster Knoten auf Layer 3) er ein erhaltenes Paket weiterleitet, dazu benutzt er einen Routing-Protokoll \cite{Tanenbaum:2003:CN}. Routing-Protokolle werden unterschieden in IGP (Interior Gateway Protokolle) und EGP (Exterior Gateway Protokolle). Erstere werden verwendet um Packte innerhalb eines Autonomen Systems zu weiterzuleiten, wohin gegen letztere für das Routing über die AS Grenzen hinaus benutzt werden.\\

% Kernproblem:
\paragraph{}
Einer der beschränkenden Faktoren für das Wachstum des Internets ist die Skalierbarkeit des Core-Routings. Diese ist vor allem dadurch gefährdet, dass die Edge-Networks in der Core-Routing-Tabelle abgebildet werden.

%Gründe des Kernproblems
\paragraph{Ursachen}
Edge-Networks müssen in die Core-Routing-Tabelle aufgenommen werden, weil Adresspräfixe nicht immer hierarchisch vergeben werden. Bei hierarchischen Vergabe erhält ein Edge-Network einen Präfix, welcher innerhalb des Präfix seines Providers liegt. Dadurch muss der Präfix dieses ENs nicht für den Internet-Core sichtbar sein, sondern nur der Präfix des Providers. \\
Die hierarchische Vergabe hat jedoch Nachteile für ein Edge-Network. So wird zum Beispiel der Wechsel des ISPs erschwert, da eine Änderung des EN-Präfixes nötig wäre. Um dies zu verhindern, benutzen die Edge-Networks eine providerunabhängige Adresse (PI). \\
Ein weiteres Problem bei der hierarchische Vergabe ist Multihoming, Verbindungen eines ENs zu mehren Providern. Mehrere Verbindung eines ENs an den Internet-Core können genutzt werden um Verkehr zu verteilen oder den Ausfall eines ISPs kompensieren. Um Erreichbarkeit durch alle ISPs zu gewährleisten, müssen Routen zum Präfix des ENs durch alle Verbindungen im gesamten Internet-Core bekannt sein, daher nutzen multihomed ENs auch PIs.

%Auswirkungen des Kernproblems
\paragraph{Folgen}
Da Egde-Networks mit providerunabhängigen Adressen in den Core-Routing-Tabelle enthalten sein müssen, hat die Zunahme dieser ENs auch ein Wachstum der Routing-Tabelle zur Folge. Bis zum Jahr 2004 wurden insgesamt des Internets ca. 63 000 Adressblöcke registriert, davon ungefähr 18 000 in den letzten 7 Jahren. Gleichzeitig enthielt die BGP Routing Tablle über 160 000 Einträge \cite{journals/ccr/MengXZHLZ04}.  Die Anzahl der Einträgen pro registrierten Adressblock nahm von 1998 mit 1,33 Einträge auf 2,54 Einträge im Jahr 2004 zu. Diese hohe Menge an Routen je Adressblock deutet darauf hin, dass viele Edge-Networks Multihoming nutzen. Dadurch wird das Wachstum der Core-Routing-Tabelle nochmals verstärkt.\\
Mit der Größe der Routing-Tabelle wächst auch die Anzahl der Updates die nötig ist um die Tabelle zu aktualisieren. In der ersten Jahreshälfte 2004 wurden 24 000 Einträge in der BGP-Tabellen entfernt und 36 000 Einträge hinzugefügt, wobei die Anzahl der erreichbaren Adressen sich nur geringfügig veränderte \cite{journals/ccr/MengXZHLZ04}. Da viele Updates von nur wenigen AS durchgeführt werden, kann es ein das schlecht konfigurierte Router unnötig viele Updates veröffentlichen.\cite{jen:2008:start}. \\

 



\section{Problemstellung}


%  kurze Begriffsklärung Einteilung
\paragraph{}
Das Internet besteht aus kleinen individuellen selbständigen Netzwerken. Eine solches Teilnetzwerk wird als AS (Autonomes System) bezeichnet, und unterliegt meist einer einzelden Institution. Ein Autonomes System welches Daten von anderen Autonomen Systemen weiterleitet wird als ISP (Internet Service Provider) bezeichnet, andernfalls wird es EN  (Edge Network) genannt \cite{Mahorta:2002:IR}. Der Internet-Core (oder auch Internet-Backbone) wird durch alle ISPs gebildet. Edge Netzwerke sind über mindestens einen ISP an das Internet angeschlossen, in dem sie über mindestens eine Verbindung zu diesem ISP verfügen. Nutzt ein Edge Netzwerk mehrer Verbindungen, oft auch zu unterschiedlichen ISPs, um Ausfallssicherheit zu bekommen oder den Netzwerkverkehr zu verteilen, wird es als Multihomed Edge Netzwerk bezeichnet. \\

% Routing: TODO: watch closely
Mit Routing wird der Vorgang bezeichnet der nötig ist um in einem Netzwerk ein Datenpaket von einem Router zum nächsten weiterzuleiten \cite{Mahorta:2002:IR}. Auf dem Weg des Paketes durch ein Netzwerk, muss an jedem Router entschieden werden, zu welchem seiner direkten Nachbarn (nächster Knoten auf Layer 3) ein erhaltenes Paket weitergeleitet werden soll. Diese Entscheidung wird an Hand der Routing Tabelle getroffen. Eine Routing Tabelle enthält Zuordnungen von Nachbarn zu IP-Addressen, die man über ihn erreichen kann \cite{Mahorta:2002:IR}. Die Einträge in der Routing Tabelle können entweder statisch fest geslegt werden oder durch einen Routing Algorithmus ermittelt werden, der als Teil eines Routing Protokolls implemntiert ist. \cite{Tanenbaum:2003:CN}. Routing-Protokolle werden unterschieden in IGP (Interior Gateway Protokolle) und EGP (Exterior Gateway Protokolle). Erstere werden verwendet um Packte innerhalb eines Autonomen Systems zu weiterzuleiten, wohin gegen letztere für das Routing über die AS Grenzen hinaus benutzt werden. \\

%TODO: hier fehlt ein zusammenführender Abschnitt. Ist das so?

% Kernproblem:
\paragraph{}
Einer der beschränkenden Faktoren für das Wachstum des Internets ist die Skalierbarkeit des Core-Routings. Diese ist vor allem dadurch gefährdet, dass die Edge-Networks in der Core-Routing-Tabelle abgebildet werden.

%Gründe des Kernproblems
\paragraph{Ursachen}
Edge-Networks müssen in die Core-Routing-Tabelle aufgenommen werden, weil Adresspräfixe in der Praxis nicht hierarchisch vergeben werden. Bei hierarchischen Vergabe erhält ein Edge-Network einen Präfix, der innerhalb des Präfix eines ISPs liegt. Dieser ISP verbindet das Edge Netzwerk mit dem Internet und leitet eingehnden Verkehr an das EN weiter. Dadurch muss der Präfix dieses ENs nicht für den restlichen Internet-Core sichtbar sein, sondern nur der Präfix des Providers.\\
Die hierarchische Vergabe führt dazu das ein Edge Netzwerk beim Wechseln das ISPs einen anderen Präfix erhält. Um die damit verbundene Umnummerierung zu vermeiden, benutzen die Edge-Networks providerunabhängige IP-Adressen (ProviderIndependet Address PI). Diese Provider unabhänigen IP-Addresen müssen in den Routing Tabelle des Internet Cores als eigene Präfixe eingetragen werden, das sonst kein Packete zu ihnen übermittelt werden können \cite{jen:2008:start}.\\
Ein weiteres Problem bei der hierarchische Vergabe stellt Multihoming dar. Um Erreichbarkeit durch alle ISPs zu gewährleisten, müssen alle Routen zum multihomed EN an im Internet sichtbar sein. Die am meisten verbreitete Möglichkeit dies zu realsieren ist die Verwendung von PIs \cite{jen:2008:start}.
%Auswirkungen des Kernproblems
\paragraph{Folgen}
Da Egde-Networks mit providerunabhängigen Adressen in den Core-Routing-Tabelle enthalten sein müssen, hat die Zunahme dieser ENs auch ein Wachstum der Routing-Tabelle zur Folge. Bis zum Jahr 2004 wurden insgesamt des Internets ca. 63 000 Adressblöcke registriert, davon ungefähr 18 000 in den letzten 7 Jahren. Gleichzeitig enthielt die BGP Routing Tablle über 160 000 Einträge \cite{journals/ccr/MengXZHLZ04}.  Die Anzahl der Einträgen pro registrierten Adressblock nahm von 1998 mit 1,33 Einträge auf 2,54 Einträge im Jahr 2004 zu. Diese hohe Menge an Routen je Adressblock deutet darauf hin, dass viele Edge-Networks Multihoming nutzen \cite{vogt:2008:six}. Dadurch wird das Wachstum der Core-Routing-Tabelle nochmals verstärkt.\\
Mit der Größe der Routing-Tabelle wächst auch die Anzahl der Updates die nötig ist um die Tabelle zu aktualisieren. In der ersten Jahreshälfte 2004 wurden 24 000 Einträge in der BGP-Tabellen entfernt und 36 000 Einträge hinzugefügt, wobei die Anzahl der erreichbaren Adressen sich nur geringfügig veränderte \cite{journals/ccr/MengXZHLZ04}. Da viele Updates von nur wenigen AS durchgeführt werden, kann es ein das schlecht konfigurierte Router unnötig viele Updates veröffentlichen.\cite{jen:2008:start}. \\

 



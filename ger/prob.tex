\section{Problemstellung}
% Routing:
\paragraph{}
Mit Routing wird der Vorgang bezeichnet, der nötig ist um ein Datenpaket in einem Netzwerk weiterzuleiten \cite{Mahorta:2002:IR}. Auf dem Weg des Paketes von der Quelle zum Ziel muss an jedem Router entschieden werden, zu welchem seiner direkten Nachbarn (nächster Knoten auf Netzwerkschicht) ein Paket weitergeleitet werden soll. Diese Entscheidung wird an Hand der Routing-Tabelle getroffen. Eine Routing-Tabelle enthält Zuordnungen zwischen Präfixen und Nachbarn, über die man den Präfix erreichen kann \cite{Mahorta:2002:IR}, dies wird auch als Routing-Policy bezeichnet \cite{hawkinson:1996:autnomousSystems}. Die Einträge in der Routing Tabelle werden entweder statisch festgelegt oder durch einen Routing-Protokoll ausgetauscht. \cite{Tanenbaum:2003:CN}. Routing-Protokolle werden unterschieden in Interior Gateway Protokolle (IGP) und Exterior Gateway Protokolle (EGP). Jenachdem ob Routing-Informationen innerhalb eines Autonomen Systems oder zwischen Autonomen Systemen ausgetauscht werden. % TODO: Am letzetn Satz könnte man noch was drehen.

% Struktur: Vom Netzwerk zu AS.
\paragraph{}
Das Internet besteht aus kleinen individuellen selbständigen Netzwerken. Ein Autonomes System (AS) wird von einer Menge solcher Netzwerke gebildet, die untereinander verbunden sind und eine gemeinsame EGP-Routing Policy besitzen \cite{hawkinson:1996:autnomousSystems}. Behandelt ein Netzwerk auschließlich Pakete deren Quelle oder Ziel es selbst ist,  wird es als Edge-Netzwerk (EN) bezeichnet. Ein Internet Service Provider ist ein AS das Verkehr an Edge-Netzwerke oder andere Autonome Systeme weiterleitet. Wenn ein ISPs keine Edge-Netzwerke bedient, handelt es sich um ein Transit-AS. Alle Autonomen Systeme die keine Pakete für andere weiterleiten werden als Stub-AS bezeichnet, und sind je nach der Anzahl ihrer Verbindungen zu ISPs in die Kategorien singelhomed und multihomed eingeteielt \cite{Mahorta:2002:IR}. Ein Edge-Netzwerk kann ein Stub-AS bilden oder im AS seines Providers enthalten sein. Die Bildung eines Autonomen Systems aus einem oder mehreren Netzwerken ist insbesondere dann nötig, wenn Routign Informationen mit anderen AS ausgetauscht werden sollen.

% Struktur: Von der Herachie zur Sammlung Autonomer Systeme.
\paragraph{}
Eine hierachische Vergabe von Präfixen führt dazu, dass die Komplexität des Routings durch die Anzahl der Subpräfixe gesteuert werden kann. Jeder Router muss ermitteln ob das Ziel in einem seiner Subpräfix liegt, anderfalls kann er das Paket entlang einer Default-Route an den übergordneten Router weiterleiten. Eine solche Struktur ist jedoch für das Internet nicht wünschenswert, denn sie erfodert eine zentrale Instanz als Wurzel und ist fehleranfällig. Das Versagen eines Routers gefärdet die Konnektivität eines gesamten Teilzweiges. Das Internet bildet eine flache Struktur, wobei die einzelenden Komponenten wieder hierachich aufgebaut sind, um die Komplexität des Routings zu begrenzen. Das theoretische Konzept der Addressvergabe im Internet sieht vor das die RIRs den großen ISPs Präfixe zuteilen, und diese anschließend die Addressen hierachisch an ihre Kunden (kleine ISPs und Edge-Netzwerke) weitergeben \cite{journals/ccr/MengXZHLZ04}. In der Routing-Perspektive wird der vorhandene Adressraum unterteilt durch in die den IPSs zu geordneten Präfixe. Es muss eine Menge von Knoten geben, die die gleiche Sicht auf die nicht hierachischen Präfixe haben. Diese Knoten sind normalerweise die EGP-Router der ISPs, sie bilden die Default-Free-Zone (DFZ). Alle Knoten der DFZ benötigen eine Route zu jedem in der DFZ bekanten Präfix. Die Router eines Stub-AS verfügen können eine Default-Route zum Internet über ihre Provider verwenden, und gehören damit meist nicht zur DFZ. Ihr Prefix wird trotzdem in der gesamten DMZ bekannt gegeben. 

% Kernproblem:
\paragraph{}
Die Skalierbarkeit des Routings zwischen Autonomen Systemen ist vor allem dadurch gefährdet, dass die Edge-Netzwerke in den Routing-Tabelle der DFZ-Router abgebildet werden.
Ein Edge-Netzwerk wird in die Core-Routing-Tabelle sichtbar, wenn es providerunabhänige Addressen (PIs) benutzt. Providerunabhänige Adressen liegen nicht innerhalb des Präfix eines Providers, dadruch muss die Route zu einen PI-Präfix einen Provider in der DFZ bekannt sein.

% Warum PIs?
\paragraph{}
Providerunabhänige Addressen werden von Edge-Netzwerken vorallem aus zwei Gründen genutzt: Sie erleichtern sie den Wechsel des Providers und sie bieten eine für das Edge-Netzwerk einfache Möglichkeit Multihoming zu realsieren. Nutzt ein Edge-Netzwerk den einen Subpräfix seines Providers, so muss es wenn es den ISP wechelt ebenfalls den Präfix wechseln \cite{jen:2008:start}. Dies erfodert die Änderung der Addresse an jedem Gerät, und ist dadurch eine sehr aufwendige Operation. \\
Viele Edge-Netzwerke nutzen Multihoming um ihre Netzwerklast auf mehrer Provider zu verteilen oder im Falle eines Verbindungsausfalls auf einen anderen Provider zurück zugreifen. Um durch alle seine Provider erreichbar zu sein, muss der Präfix des ENs in der gesamten DFZ sichtbar sein. Kein ISP kann eine multiomed Edge-Netzwerk in seinen eigenen Präfix einglidern kann \cite{jen:2008:start}, es sei den es wird eine Struktur geschaffen die diese Eingliederung für andere Endgeräte transperent macht.

% Auswirkungen des Kernproblems
\paragraph{}
Bis zum Jahr 2004 wurden insgesamt des Internets ca. 63 000 Adressblöcke registriert, davon ungefähr 18 000 in den letzten 7 Jahren. Im gleichen Zeitraum durch geführte Messungen am BGP-Protokoll, dem meist verwendeten Exterior Gateway Protokoll zeigten, dass die Routing-Tabellen der DMZ ca. 160 000 Einträge umfassen \cite{journals/ccr/MengXZHLZ04}. Dazu wurden die Routing-Tabellen von verschiedener DMZ-Router aus ausgewählten Autonomen Systemen zusammen geführt und ausgewertet. Die Anzahl der Einträgen pro registrierten Adressblock nahm von 1998 mit 1,33 Einträge auf 2,54 Einträge im Jahr 2004 zu \cite{journals/ccr/MengXZHLZ04}. Diese hohe Menge an Routen je Adressblock deutet darauf hin, dass viele Edge-Networks Multihoming nutzen \cite{huston:2001:analyzing}.\\

Eine weitere Faktor der zum Wachstum der Routing-Tabelle beiträgt ist die unsaubere Allokation von Präfixen. Idealerweise sollte ein AS über genau einen Präfix haben, der die Teilnetzwerke enthält \cite{hawkinson:1996:autnomousSystems}. In der Praxis verfügen jedoch viele Autonome Systeme über mehrer Präfixe, die sie alle in der DFZ bekannt geben. Außerdem werden Präfixe, die eigentlich zusammengefasst werdee können, über verschiedene Autonome Systeme verteilet oder Subpräfixe werden einzelnd propagandiert, so dass die Anzahl der Routen in der DFZ steigt. \\

Mit der Größe der Routing-Tabelle wächst auch die Anzahl der Updates die nötig ist um die Tabelle zu aktualisieren. In der ersten Jahreshälfte 2004 wurden in der Routing Tabelle  24 000 Einträge in der BGP-Tabellen entfernt und 36 000 Einträge hinzugefügt, wobei die Anzahl der erreichbaren Adressen sich nur geringfügig veränderte \cite{journals/ccr/MengXZHLZ04}. Dies impliziert das nicht nur das Suchen in der Tabelle sondern auch das Pflegen der Tabelle aufwendiger wird und die DFZ-Router belastet.

% Locator <> Identifier Doppelbedeutung
\paragraph{}
Ein weiteres Problem ist, dass eine Unicast-Adresse in der Praxis eine überladene Bedeutung besitzt. Zum einen identifiziert sie ein einzelnes Interface zu einem bestimmten Zeitpunkt, gleichzeitig dient sie aber auch dazu ein Gerät zu lokalisieren. Da eine Verbindung zwischen zwei Geräten an die Addressen der jeweiligen Gesprächspartner gebunden ist, bedeutet ein Wechsel des Netzes auch den Abbruch der Verbindung.
%TODO: Frage: was heißt das für Multihoming und Providerwechsel.



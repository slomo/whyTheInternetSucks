\section{Problemstellung}
% Routing:
\paragraph{}
Mit Routing wird der Vorgang bezeichnet, der nötig ist, um ein Datenpaket in einem Netzwerk weiterzuleiten \cite{Mahorta:2002:IR}. Auf dem Weg des Paketes von der Quelle zum Ziel muss an jedem Router entschieden werden, zu welchem seiner direkten Nachbarn (nächster Knoten auf der Netzwerkschicht) ein Paket weitergeleitet werden soll. Diese Entscheidung wird anhand der Routing-Tabelle getroffen. Eine Routing-Tabelle enthält Zuordnungen zwischen Präfixen und Nachbarn, über die man den Präfix erreichen kann \cite{Mahorta:2002:IR}. Erhält der Router ein Paket, so geht er die Routing-Tabelle durch und ermittelt die Menge der Präfixe in denen die Zieladresse enthalten ist \cite{Tanenbaum:2003:CN}. Anschließend leitet er das Paket an den Nachbarn weiter, der dem längsten dieser Präfixe zugeordnet ist. Die Einträge in der Routing-Tabelle werden entweder statisch festgelegt oder durch ein Routing-Protokoll ausgetauscht \cite{Mahorta:2002:IR}. Routing-Protokolle werden unterschieden in Interior Gateway Protokolle (IGP) und Exterior Gateway Protokolle (EGP) \cite{Tanenbaum:2003:CN}. Mit EGPs werden Routing-Informationen zwischen Autonomen Systemen ausgetauscht, während IGPs innerhalb eines AS verwendet werden.  % TODO: IGPs ?

% Struktur: Vom Netzwerk zu AS.
\paragraph{}
Das Internet besteht aus kleinen, individuellen, selbständigen Netzwerken. Ein Autonomes System (AS) wird von einer Menge solcher Netzwerke gebildet, die untereinander verbunden sind und eine gemeinsame EGP Routing-Policy besitzen \cite{hawkinson:1996:autnomousSystems}. Behandelt ein Netzwerk ausschließlich Pakete, deren Quelle oder Ziel es selbst ist,  wird es als Edge-Netzwerk (EN) bezeichnet. Ein Internet Service Provider ist ein AS, das Verkehr an Edge-Netzwerke oder andere Autonome Systeme weiterleitet. Wenn ein ISP keine Edge-Netzwerke bedient, handelt es sich um ein Transit-AS. Alle Autonomen Systeme, die keine Pakete für andere weiterleiten, werden als Stub-AS bezeichnet. Sie werden, je nach der Anzahl ihrer Verbindungen zu ISPs, in die Kategorien singelhomed und multihomed eingeteilt \cite{Mahorta:2002:IR}. Ein Edge-Netzwerk kann ein eigenständiges Stub-AS bilden oder Teil des Autonomen Systems seines Providers sein. Die Bildung eines Autonomen Systems aus einem oder mehreren Netzwerken ist insbesondere dann nötig, wenn Routing-Informationen mit anderen Autonomen Systemen ausgetauscht werden sollen \cite{hawkinson:1996:autnomousSystems}.

% Struktur: Von der Herachie zur Sammlung Autonomer Systeme.
\paragraph{}
Eine strikt hierarchische Vergabe von Präfixen führt dazu, dass die Komplexität des Routings durch die Anzahl der Subpräfixe gesteuert werden kann. Jeder Router muss ermitteln, ob das Ziel in einem seiner Subpräfixe liegt, andernfalls kann er das Paket entlang einer Default-Route an den übergeordneten Router weiterleiten. Der Router an der Wurzel dieser Baumstruktur verfügt über keine Default-Route und muss daher Pakete zu unbekannten Präfixen verwerfen. Eine solche Hierarchie ist jedoch für das Internet nicht wünschenswert, denn sie erfordert eine zentrale Instanze als Wurzel mehrerer Teilbäume. Dieser Knoten kann hohen Lasten ausgesetzt sein und bildet eine kritische Fehlerquelle für seinen gesamten Unterbaum. Das Internet bildet eine flache Struktur auf AS-Ebene, wobei die einzelnen Komponenten wieder hierarchisch aufgebaut sind, um die Komplexität des Routing-Systems zu begrenzen. Das Konzept der Adressvergabe im Internet sieht vor, dass die RIRs den großen ISPs Präfixe zuteilen, und diese anschließend die Adressen hierarchisch an ihre Kunden (kleine ISPs und Edge-Netzwerke) weitergeben \cite{ripe:ipv4}. In der Routing-Perspektive wird der vorhandene Adressraum unterteilt in die den IPSs zugeordneten Präfixe. \\

% TODO: Angeblich kann man das hier besser machen
Es gibt Knoten im Internet, welche die gleiche Sicht auf alle nichthierarchischen Präfixe benötigen. Diese Knoten sind normalerweise die EGP-Router der ISPs. Sie besitzen keine Default-Routen und bilden daher die Default-Free-Zone (DFZ). Alle Knoten der DFZ benötigen eine Route zu jedem in der DFZ bekannten Präfix. Die Router eines Stub-AS können eine Default-Route zum Internet über ihre Provider verwenden und gehören damit meist nicht zur DFZ. Ihr Präfix wird trotzdem in der gesamten DFZ bekanntgegeben. 

% Kernproblem:
\paragraph{}
Die Skalierbarkeit des Routings zwischen Autonomen Systemen ist vor allem dadurch gefährdet, dass die Edge-Netzwerke in den Routing-Tabellen der DFZ-Router abgebildet werden.
Ein Edge-Netzwerk kann im Präfix seines Providers aggregiert sein oder eine eigene Route besitzen. Um letzteres zu ermöglichen nutzt das Edge-Netzwerk  providerunabhängige Adressen (PI) oder deaggregierte Adressen seines Providers \cite{jen:2008:start}. Ein providerunabhängiger Präfix wird direkt von der RIR vergeben, und liegt nicht innerhalb des Präfix eines Providers. Ist der Präfix eines Edge-Netzwerkes nicht von seinem Provider aggregiert, muss der Pfad zum EN über seinen Provider in allen Routing-Tabellen der DFZ enthalten sein.

% Warum PIs?
\paragraph{}
Providerunabhängige Adressen werden von Edge-Netzwerken vor allem aus zwei Gründen genutzt: Sie erleichtern den Wechsel des Providers und sie bieten eine für das Edge-Netzwerk einfache Möglichkeit Multihoming zu realisieren. Nutzt ein Kunde den Subpräfix seines Providers, so erhält er beim Wechsel des ISPs einen anderen Präfix. \cite{jen:2008:start}. Dies erfordert die Änderung der IP-Adresse an jedem Gerät und ist dadurch eine aufwendige Operation. \\
Viele Edge-Netzwerke nutzen Multihoming, um ihre Netzwerklast auf mehrere Provider zu verteilen oder im Falle eines Verbindungsausfalls auf einen anderen Provider zurückzugreifen. Um über alle seine angebundenen Provider adresstransparent erreichbar zu sein, muss der Präfix des ENs in der gesamten DFZ sichtbar sein. Kein ISP kann ein multihomed Edge-Netzwerk in seinen eigenen Präfix eingliedern \cite{jen:2008:start}, es sei denn es wird eine Struktur geschaffen, die diese Eingliederung für andere Endgeräte transparent macht.

% Auswirkungen des Kernproblems
\paragraph{}
Bis zum Jahr 2004 wurden insgesamt ca. 63 000 IP-Adressblöcke registriert, davon ungefähr 18 000 in den letzten 7 Jahren. Im gleichen Zeitraum durchgeführte Messungen am meist verwendeten Exterior Gateway Protokoll BGP zeigten, dass die Routing-Tabellen der DFZ ca. 160 000 Einträge umfassen \cite{journals/ccr/MengXZHLZ04}. Dazu wurden die Routing-Tabellen verschiedener DFZ-Router aus ausgewählten Autonomen Systemen zusammengeführt und ausgewertet. Die Anzahl der Einträge pro registriertem IP-Adressblock nahm von 1998 mit 1,33 Einträge auf 2,54 Einträge im Jahr 2004 zu \cite{journals/ccr/MengXZHLZ04}. Diese hohe Menge an Routen je Adressblock deutet darauf hin, dass viele Edge-Netzwerke Multihoming nutzen \cite{huston:2001:analyzing}.\\

% ASe verhalten sich nicht koscha
Ein weiterer Wachstumfaktor für die Routing-Tabellen ist die unsaubere Allokation von Präfixen. Idealerweise sollte ein AS über genau einen Präfix verfügen, der alle Netzwerke des AS enthält \cite{hawkinson:1996:autnomousSystems}. Außerdem werden Präfixe, die eigentlich zusammengefasst werden können, über verschiedene Autonome Systeme verteilt oder Subpräfixe werden deaggregiert, so dass die Anzahl der Routen in der DFZ steigt \cite{journals/ccr/MengXZHLZ04}. \\

Mit der Größe der Routing-Tabelle wächst auch die Anzahl der Updates, die nötig sind um die Tabelle zu aktualisieren. In der ersten Jahreshälfte 2004 wurden 24 000 Einträge in der BGP-Tabellen entfernt und 36 000 Einträge hinzugefügt, wobei die Anzahl der erreichbaren Adressen sich nur geringfügig veränderte \cite{journals/ccr/MengXZHLZ04}. Dies impliziert, dass nicht nur das Suchen in der Tabelle sondern auch das Pflegen der Tabelle aufwendiger wird und die DFZ-Router zusätzlich belastet.

% Locator <> Identifier Doppelbedeutung
\paragraph{} 
Ein weiteres Problem ist, dass eine Unicast-Adresse in der Praxis eine überladene Bedeutung besitzt. Sie wird als Identifier genutzt, da sie einen Socket während eines Kommunikationsvorganges identifiziert. Gleichzeitig dient sie dem Routing-System als Locator. Ändert sich also der Präfix bezüglich eines Gerätes, erzwingt dies eine Unterbrechung der Socket-Kommunikation. Dies erschwert die Implementierung von MobilIP und Multihoming ohne PI.



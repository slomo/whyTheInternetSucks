%Einleitungskapitel
\section{Einleitung}

%TODO: Kompletter rewrite

\paragraph{Internet}
Das Internet besteht aus kleineren individuellen selbständigen Einheiten. Diese Einheiten sind mehr oder weniger homogene Netzwerke welche von einer Institution verwaltet werden. Ein solches Netzwerk wird als Autonomes System (AS) bezeichnet. Autonome Systeme werden in verschidene Kategoreien eingeteilet, je nachdem wie sie mit anderen ASs verbunden sind, bzw. wie sie mit deren Netzverkehr üblicherweise verfahren. Edgenetworks (EN), empfangen und senden nur Daten die für sie selbst bestimmt sind, sie werden auch als leafnetworks oder stubnetworks bezeichnet. Netzwerke die durchgehende Verkehr (transittraffic) weiterleiten, sind üblichwerweise ISPs sie verbinden edgenetworks, werden kategorsiert nach räumlicher Ausdehnung (lokal,national oder global) und bilden alle zusammen den sogennaten Internetcore. Verbindungen zwischen zwei AS (vorallem wenn sie der gleichen Kategorie angehören) werden auch als Peeringpoint bezeichnet. Ein AS wird durch eine eindeutige Nummer identifiziert, welches ihm durch die RIPE zugeordnet wird.

\paragraph{Routing}
Mit Routing wird eine Menge von Vorgängen bezeichnet die nötig ist um in einem Netzwerk ein Datenpaket von einem Router zum nächsten weiterzuleiten, bis es letztendlich sein Ziel erreicht \cite{Mahorta:2002:IR}. Auf dem Weg, den das Paket durch ein Netzwerk nimmt, muss an jedem Router entschieden werden, zu welchem seiner direkten Nachbarn (nächster Knoten auf Layer 3) er ein erhaltenes Paket weiterleitet, dazu benutzt er einen Routing-Algorithmus \cite{Tanenbaum:2003:CN}. Routing-Algorithmen werden unterschieden in IGP (Interior Gateway Protokolle) und EGP (Exterior Gateway Protokolle). Erstere werden verwendet um Packte innerhalb eines Autonomen Systems zu weiterzuleiten, wohin gegen letztere für das Routing über die AS Grenzen hinaus benutzt werden. 



%Einleitungskapitel
\section{Einleitung}
%TODO: kann hier mehr
Durch das starke Wachstum der letzten Jahre erreicht das Internet die technischen Grenzen der aktuell verwendeten Protokolle. Die limitirenden Faktoren sind die Addresslänge und die Komlexität des Routing-Algorithmus. Während mit IPv6 das Problem der Addresslänge weitgehend gelöst wurde, bleibt noch offen wie die Routing-Struktur für die wachsenden Teilnehmerzahlen verändert werden soll. \\

Während der letzten Jahre wuchs die Größe der Routing-Tabelle in der DFZ (Default-Free-Zone) mehr als linear zur Anzahl der verbundenen Systeme und gefährdet damit die Sklaierbakeit des Routings \cite{Huston:2003:BGP}. \\

Diese Arbeit soll aufzeigen welche aktuellen Entwicklungen das Internet-Routing gefährden, wo das prinzipielle Problem liegt und welche möglichen Lösungsansätze innerhalb der zuständigen Grämien diskutierte werden. \\

Der erste Abschnitt soll in das Thema einleiten und die nötigen Begriffe erläutern, um im darauf folgenden Teil das Problem, sowie dessen Ursachen und Auswirken genauer zu erläutern. Zwei prinzipielle Lösungsmöglichkeiten werden im 3 Abschnitt aufgzeigt, wobei der Seperationsansatz genauer diskutiert wird. Im letzten Abschnitt werden zuvor gewonnen Erkentnisse und bewertet.

 

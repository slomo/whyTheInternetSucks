%Einleitungskapitel
\section{Einleitung}

\paragraph{Internet}
Das Internet verbindet kleinere Netze zu einem Größeren ganzen, um den Datentransfer zwischen ihnen zu ermöglichen. Die Bestandteile der Gesamtstruktur stehen jeweils unter der Verwaltung einer einzelenden Institution, man bezeichnet sie als Autonomes System (AS). Autonome Systeme können unterteilet werden in solche die Daten von anderen AS annehmen und weiterleiten, sie werden auch als ISP bezeichnet, und solche die nur ihre eigenen Daten empfangen oder senden. Alles ASs können wohl unterschiden werden anhand ihrer AS-Nummer, welche ihnen von der RIPE zugeordnet wird.

\paragraph{Routing}
Mit Routing wird eine Menge von Vorgängen bezeichnet die benötigt werden um in einem Netzwerk ein Paket von einem Router zum nächsten weiterzuleiten, bis es letztendlich sein Ziel erreicht \cite{Mahorta:2002:IR}. Auf dem Weg den das Paket durch ein Netzwerk nimmt muss an jedem Router entschieden werden, zu welchem seiner direkten Nachbarn (nächster Knoten auf Layer 3) er ein erhaltenes Paket weiterleitet, dazu benutzt er einen Routing-Algorithmus \cite{Tanenbaum:2003:CN}. Routing-Algorithmen werden unterschieden in IGP (Interior Gateway Protokolle) und EGP (Exterior Gateway Protokolle). Erstere werden verwendet um Packte innerhalb eines Autonomen Systems zu weiterzuleiten, wohin gegen letztere für das Routing über die AS Grenzen hinaus benutzt werden.



%Einleitungskapitel
\section{Einleitung}
%TODO: kann hier mehr
Durch das starke Wachstum der letzten Jahre nährt sich Internet den technischen Grenzen der aktuell in Verwendung befindlichen Protokolle. Die limitirenden Faktoren sind die Addresslänge und die Komlexität des Routing-Algorithmus. Während mit IPv6 das Problem der Addresslänge weitgehend gelöst wurde, bleibt noch offen wie die Routing-Struktur für die wachsenden Teilnehmerzahlen verändert werden soll. \\

Während der letzten Jahre wuchs die Größe der Routing-Tabelle in der DFZ (Default-Free-Zone) mehr als linear zur Anzahl der verbundenen Systeme und gefährdet damit die Sklaierbakeit des Routings. \\

Diese Arbeit soll aufzeigen welche aktuellen Entwicklungen das Internet-Routing gefährden, wo das prinzipiellen Problem liegt und welche möglichen Lösungsansätze innerhalb der zuständigen Grämien diskutierte werden. \\

Der Erste Abschnitt soll in das Thema einleiten und die nötigen Begriffe erläutern, um darauf folgend Abschnitt das Problem, sowie dessen Ursachen und Auswirken genauer zu erläutern. Zwei prinzipielle Lösungsmöglichkeiten werden im 3 Abschnitt aufgzeigt, wobei der Seperationsansatz tiefergehnd diskutiert wird. Im letzten Abschnitt werden die Lösungen zusammengefasst und soweit möglich bewertet.

 

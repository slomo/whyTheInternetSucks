%Einleitungskapitel
\section{Einleitung}
\paragraph{Die Struktur des Internets}
Vor Beginn meines Studiums stellt ich mir das Internet als eine Menge 
von routern vor, welche eher un struktiriert untereineander verbunden 
sind. Es war offensichtlich das der einzelnden Computer nicht die 
genaue Struktur des Internets kannte, jedoch schien der Router an 
welchen ich meine Daten (bzw. Pakete) weiterleitet genau zu wissen, wo 
hi 
%% weritermachen 
\paragraph{Bauteile des Internet}
\subparagraph{Autonomous Systems}
Das Internet 


Before starting my studies I tough of the Internet as a large crowd of 
Routers, connect with each other without any specific structure, it was 
obvious that not ever single computer in the net could have a plan or 
some kind of map for the whole Internet, but the routers seemed to have 
all the same understanding of the net. A transmission was done by 
sending the data, also referred by the term package, to your nearest 
router, described by your default route, who selected which way to take 
through the entire net. \\
During my first year I my view to the Internet changed from this 
cloud perspective to a more hierarchical perspective, as i was introduced 
in managing a Router at spline. Also this router hat a default route 
and routes, to other routers he know, which are the default router for 
their subnet. Thinking of the Internet as such a structured tree 
implies that there exists one rootnode which does not go along with any 
of terms usually used by people to describe the net, for example 
fault tolerant or equal. \\
The truth about the structure of the Internet is that it is as well 
something in between that tow approaches and also something completely 
different. 
\paragraph{Parts of the Internet}
\subparagraph{Autonomous Systems}
The Internet is a net build out of nets. Each of these smaller net 
builds an autonomous system (AS). Each of these is controlled by single 
operator, often a company, a university or any other kind of 
organistation. ASs can be identified via an AS-Number, normally written 
as AS<number>.
%% add more here 
\subparagraph{Interior Gateway Protocol}
If a machine inside such a autonomous system wants to reach an other 
machine it forwards the packet to a router, assuming that the receiver 
is not part of the same subnet. If the destination is within the same AS 
the router uses the IGP (Interior Gateway Protocol) to discover the 
shortest path to it. IGP is used to direct traffic, which goes from 
one location in the AS to an other. These location can be systems in 
the AS, as well as exit or entry points to the AS.

%Einleitungskapitel
\section{Einleitung}

\paragraph{}
Durch die starke Zunahme an Teilnehmern in den letzten Jahren erreicht das Internet seine technischen Grenzen auf der Netzwerkschicht. Probleme zeigen sich unter anderem bei der Anzahl an verfügbaren Adressen, der Mobilität von Endgeräten, der Sicherheit des Gesamtsystems und den Routing-Protokollen. Durch IPv6 wurde für die Anzahl der Adressen und Teilaspekte der Mobilität eine Lösung gefunden. Auch das Thema Sicherheit, als begleitender Prozess jeder Entwicklung, gewann in den letzten Jahren mehr Aufmerksamkeit. Der Handlungsbedarf beim Internet-Routing wurde schon vor einiger Zeit erkannt \cite{deering:1996:map}. Der hohe Verteilungsgrad, die hohe Anzahl an Akteuren und die Einflüsse auf andere Subsysteme des Internets erschwerten die Suche nach neuen Lösungen für das Routing-System. Trotzdem gibt es interessante Lösungsvorschläge, um auch diesen Grundbestandteil des Internets auf zukünftige Anforderungen vorzubereiten. \\
 
Während der letzten Jahre wuchs die Größe der Routing-Tabelle in der Default-Free-Zone (DFZ) exponentiell an, trotz nur linearem Wachstum der verfügbaren Adressen\cite{huston:2001:analyzing}. Da das Routing-System mit wachsender Netzwerkgröße nicht skaliert, ist seine Funktion gefährdet \cite{jen:2008:start}. Einen der größten Faktoren für diesen Anstieg stellt redundante Anbindung von Edge-Netzwerken an das Internet (Multihoming) dar. Neben der steigenden Tabellegröße gefährdet auch die daraus resultierende hohe Frequenz an Updates das Routing-System. Alle bisherigen Lösungsvorschläge fallen in zwei grundsätzliche Kategorien: Separation oder Elimination \cite{jen:2008:start}. \\

Diese Arbeit soll aufzeigen, welche aktuellen Entwicklungen die Skalierbarkeit des Internet-Routing gefährden und die Gründe dafür untersuchen. Es werden Lösungsansätze vorgestellt, die zur Zeit innerhalb von Forschung und Standardisierung diskutiert werden. \\

Der erste Abschnitt soll in das Thema einleiten und die nötigen Begriffe erläutern, um anschließend Probleme sowie deren Ursachen und Auswirkungen genauer zu untersuchen. Zwei prinzipielle Lösungsmöglichkeiten werden im dritten Abschnitt aufgezeigt, wobei Ansätze zur Separation ausführlich diskutiert werden. Schlussendlich werden zuvor gewonnene Erkenntnisse zusammengefasst und bewertet.

 

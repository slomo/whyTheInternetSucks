\section{Bewertung}
Im Rahmen Internet Enigneering Task Force (IETF) werden bereits die einzelden Lösungsansätze, die in dieser Arbeit abgesprochen werden, ausgerarbeiet und diskutiert. Aktuell gibt es sowohl zu LISP als auch zu shim6 aktive Arbeitsgruppen innerhalb der IETF \cite{ietf:groups}. \\

Bei dem Eliminationansatz shim6 existiert bereits der RFC 5533 und damit eine gültige Spezifikation des Protokolls \cite{nordmark:2009:RFC5533},\cite{ietf:documents}. LISP scheint zur Zeit der einzige diskutierte Separationsansatz zu sein. Zu anderen Standarts, wie zum Beispiel GSE, Six/One, TRRP oder APT finden sich im Internet-Archiev der ITEF keine veröffentlichten Entwürfe. Der Entwurf zu einem LISP-RFC \cite{farinacci:2009:LISP} existiert zur Zeit in der 5 Revision. Er wurde jedoch noch nicht dem Zuständigen Area Director der ITEF vorgelegt und auch ein Antrag auf Veröffentlichung existiert nicht \cite{ietf:documents}.  Multicast-Erweiterungen für LISP und die Mappingsysteme befinden sich in einem vergleichbaren oder frühren Stadium. \\

Eine Diskusionen der Lösungsvorschläge findet innerhalb der Routing Reasearch Group (RRG) statt, die Teil der Group Internet Resarch Task Force ist. \cite{irtf:rrg}. Ein mögliches Ergebnis der RRG könnte die Empfehlung einer Lösung an die IETF sein. Letztendlich liegt es jedoch in der Zuständigkeit der AS-Betreiber ein Protokolle auzuwählen und umzusetzen. \\

%TODO: mehr
Obwohl beide Lösungsansätze das prinizpielle Problem der Routing-Tabelle in der DMZ lösen \cite{jen:2008:start}.Beide bieten auch Lösungen für Probleme im Zumsammenhang mit MobilIP \cite{farinacci:2009:LISP},\cite{nordmark:2009:RFC5533}. Es gibt aber eine Reihe von Gründen die für den Separationsansatz sprechen. \\ 

Das Verbieten von PIs schränk die Freiheiten der Edge-Betreiber ein. % TODO: ist das der Topgrund??  
Um das Locator-Identfier-Problem zu lösen, speichern beide Ansätze eine Art Status. Beim Eleminationsansatz mit shim6, werden sich änderne Addressen anderer Geräte beim endgerät gespeichert. Die Einführung eines Status auf Netzwerkschicht wieder spricht dem Prinzip eines statusfreien Netzes. Da die Speicherung des Status bei shim6 dezentralsiert statt findet, erschwert dies die Suche nach Unregelmäßigkeiten und ist für den Betrachter nicht transparent. Der Seperationsansatz verschiebt diesen Status in das Mappingsystem. \\
Ein weitere Nachteil der Elemination ist, dass ihre grundlegenden Techniken auf IPv6 basieren. Erst durch IPv6 kann eine Netzwerkschnittstelle per Entwurf mit mehrer Adressen versehen werden. Auch shim6 kann nur IPv6 um die gewünschten Funktionen erweitern. Der Sperationsansatz bietet hier eine generischere Lösung, insbesondere bei der Verwendung von LISP. Durch den Tunneling-Mechanismus müssen keine Annahmen über die Netzwerkebene getroffen werden. Dies unterscheidet LISP auch von Six/One, welches auf die IPv6-Extension-Header angewiesen ist. Eine IPv6-basierende Lösung hat vor allem den Nachteil, dass eine flächendenke Nutzung von IPv6 nicht zeitnahe zu erwarten ist. \\
Der Separationansatz erschafft eine Trennschicht zwischen DMZ und Edge-Netzwerken. Dies Trennschicht kann in Zukunft genutzt werden, um Änderungen am Internet Protokoll unabhänig vom jeweils anderen Teil durchführenzukönnen \cite{jen:2008:start}. \\
Die Einführung der RLOCs bietet die Möglichkeit ohne historische Lasten die Präfix in der DMZ neuzuordnen, um eine noch kleinere Routing-Tabelle zu erhalten. \\

Der weitere Fortgang der Diskussion bleibt abzuwarten. Es kann aber angenommen werden das mit der flächendeckenden Durchsetzung von IPv6 und der damit verbundenen höhren Anzahl an Präfixen das Problem sich weiter verschärfen wird. Dies könnte zusätlichen Druck auf die Akteure ausüben, und zu einer schellen Entscheidung und Umsetzung führen.




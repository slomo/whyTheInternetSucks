\section{Bewertung}
Im Rahmen Internet Enigneering Task Force (ITEF) wurden bereits alle Lösungsansätze diskutiert, die in dieser Arbeit abgesprochen werden. Aktuell gibt es sowohl zu LISP als auch zu shim6 aktive Arbeitsgruppen innerhalb der ITEF \cite{ietf:groups}. Damit werden Protokolle diskutiert, mit denen jewiels einer der beiden grundsätlichen Annsätze möglich wird. \\

Für shim6 existiert bereits der RFC 5533 und damit eine gültige Spezifikation des Protokolls \cite{nordmark:2009:RFC5533} \cite{ietf:documents}. Da Shim6 auf Mechnismen von IPv6 aufbaut, ist ein Umstieg auf das neue Netzwerkprtokoll nötig. \\

LISP scheint zur Zeit der einzige diskutierte Separationsansatz zu sein. Zu anderen Standarts, wie zum Beispiel GSE, Six/One, TRRP oder APT finden sich im Internet-Archiev der ITEF keine veröffentlichten Entwürfe. Der Entwurf zu einem LISP-RFC \cite{farinacci:2009:LISP} existiert zur Zeit in der 5 Revision, wurde jedoch weder dem Zuständigen Area Advisor der ITEF vorgelegt, noch ein Antrag auf Veröffentlichung gestellt \cite{ietf:documents}. Erweiterungen, um etwa LISP mit Multicast-Fähigkeiten auszustatten, und die Mappingsysteme befinden sich in einem vergleichbaren oder frühren Stadium. \\

Damit läuft die Debatte um eine Lösung, eine endgültige Entscheidung wurde aktuell noch nicht getroffen. Ein Zeitplan wann eine solche Entscheidung der zuständigen ITEF-Gruppe zu erwarten ist, konnte ich während meiner Recherche auf der Website nicht finden. \\

Obwohl beide Lösungsansätze das prinizpielle Problem der Routing-Tabelle in der DMZ lösen \cite{jen:2008:start}, ist meiner Meinung nach der Sperationansatz überlegen. Zum einen wurde ich überzeugt, dass das verbieten der providerunabhänigen Adressen die Freiheit der Edge-Netzwerkbetreiber einschränkt, und auch eine steigende Komplexität bei den Endgeräten hervor ruft. Zum anderen stellt der Sperationsansatz auch eine Lösung für die Mobiltät von Endgeräten dar und erschafft eine Trennschicht zwischen DMZ und Edge-Netzwerken. Dies Trennschicht kann in Zukunft genutzt werden, um in einem der entstehenden Teile des Internets Protolländerungen unabhänig von der anderen umzusetzen \cite{jen:2008:start}. Zu dem kann die Verteilung der RLOCs genutzt werden um eine bessere Verteilung der Addressen für Provider zu erreichen und somit weniger deaggregirte Prefix entstehen zu lassen.

Der weitere Fortgang der Diskussion bleibt abzuwarten. Es kann aber angenommen werden das mit der flächendeckenden Durchsetzung von IPv6 und der damit verbundenen höhren Anzahl an Präfixen das Problem sich weiter verschärfen wird. Dies könnte zusätlichen Druck auf die Akteure ausüben, und zu einer schellen Entscheidung und Umsetzung führen.




\section{Zusammenfassung und Bewertung}
\paragraph{}
Im Rahmen Internet Enigneering Task Force (IETF) werden bereits die einzelden Lösungsansätze, die in dieser Arbeit abgesprochen werden, ausgerarbeiet und diskutiert. Aktuell gibt es sowohl zu LISP als auch zu shim6 aktive Arbeitsgruppen innerhalb der IETF \cite{ietf:groups}.

\paragraph{}
Bei der Elimination basierend auf shim6 existiert bereits der RFC 5533 und damit eine gültige Spezifikation des Protokolls \cite{nordmark:2009:RFC5533},\cite{ietf:documents}. LISP scheint zur Zeit der einzige diskutierte Separationsansatz zu sein. Zu anderen Standarts, wie zum Beispiel GSE, Six/One, TRRP oder APT finden sich im Internet-Archiev der ITEF keine veröffentlichten Entwürfe. Der Entwurf zu einem LISP-RFC \cite{farinacci:2009:LISP} existiert zur Zeit in der 5 Revision. Er wurde jedoch noch nicht dem Zuständigen Area Director der ITEF vorgelegt und auch ein Antrag auf Veröffentlichung existiert nicht \cite{ietf:documents}.  Multicast-Erweiterungen für LISP und die Mappingsysteme befinden sich in einem vergleichbaren oder frühren Stadium. 

\paragraph{}
Eine Diskusionen der Lösungsvorschläge findet innerhalb der Routing Reasearch Group (RRG) statt, die Teil der Group Internet Resarch Task Force ist. \cite{irtf:rrg}. Ein mögliches Ergebnis der RRG könnte die Empfehlung einer Lösung an die IETF sein. Letztendlich liegt es jedoch in der Zuständigkeit der AS-Betreiber ein Protokolle auzuwählen und umzusetzen.

\paragraph{}
Sowohl die Separation und als Elimination lösen die diskutierten Problem in der Routing-Tabelle der DMZ \cite{jen:2008:start}. Beide beinhalten Konzepte die für Mobile-IP genutzt werden können \cite{farinacci:2009:LISP},\cite{nordmark:2009:RFC5533}. Es gibt aber eine Reihe von Gründen, die für eine Umsetzung des Separationsansatzes sprechen. 

\paragraph{}
Um das Locator-Identfier-Problem zu lösen, speichern beide Ansätze einen Zustand. Beim Eleminationsansatz mit shim6, werden die erreichbare Addressen der Kommunikationspartner beim Endgerät gespeichert. Diese Speicherung auf Netzwerkschicht widerspricht dem Prinzip eines zustandsfreien Netzes. Der dezentral gespeicherte Zustand, erschwert die Suche nach Unregelmäßigkeiten und ist für Betrachter nicht transparent. Der Seperationsansatz speichert die Zuordnung von EIDs und RLOCs zentral im Mapping-System und bietet dem gesamten Internet die selbe Sicht auf den Status. \\
Ein weitere Nachteil der Elemination ist, dass ihre grundlegenden Techniken auf IPv6 basieren. Erst durch IPv6 kann eine Netzwerkschnittstelle per Entwurf mit mehrer Adressen versehen werden. Auch shim6 stellt eine Erweiterung für IPv6 dar. Der Sperationsansatz bietet hier eine generischere Lösung, insbesondere bei der Verwendung von LISP. Durch den Tunneling-Mechanismus müssen keine Annahmen über die Netzwerkebene getroffen werden. Dies unterscheidet LISP auch von Six/One, welches auf die IPv6-Extension-Header nutzt ist. IPv6-basierende Lösungen habe vor allem den Nachteil, dass eine flächendenke Nutzung von IPv6 abgewartet werden muss. \\
Der Separationansatz erschafft eine Trennschicht zwischen der DFZ und Edge-Netzwerken. Dies Trennschicht kann in Zukunft genutzt werden, um Änderungen auf der Netzwerkschicht unabhänig vom jeweils anderen Teil durchführen zu können \cite{jen:2008:start}. \\
Die Einführung der RLOCs bietet die Möglichkeit ohne historische Lasten die Präfix in der DMZ neuzuordnen, um eine noch kleinere Routing-Tabelle zu erhalten. \\
Die Einführung von Eliminationsprotkollen wie shim6 erfodert die Mitarbeit vieler Netzwerkbetreiber, wobei der Großteild er Kosten beim Umbau von kleineren Edge-Netzwerken enstehen wird. Außerdem werden die Edge-Netzwerkbetreiber gezwungen auf die Vorteile ihrer providerunabhänige Addressen verzichten, was sie im gewissen Maße als Entmüdigung warhnehmen könnten.

\paragraph{}
Der weitere Fortgang der Diskussion bleibt abzuwarten. Es zu erwarten, dass spätestens mit der flächendeckenden Durchsetzung von IPv6 und der damit verbundenen höhren Anzahl an Präfixen das Problem sich weiter verschärfen wird. Vorrallem da auf anfängliche Pläne für IPv6 keine PIs vorzusehen verzichtet wurde. Dies könnte zusätlichen Druck auf die Akteure ausüben, und zu einer schelleren Entscheidung und Umsetzung führen.

\section{Zusammenfassung und Bewertung}
\paragraph{}
Im Rahmen der Internet Enigneering Task Force (IETF) wurden bereits die einzelden Lösungsansätze, die in dieser Arbeit angesprochen werden, ausgearbeitet und diskutiert. Aktuell gibt es sowohl zu LISP als auch zu shim6 aktive Arbeitsgruppen innerhalb der IETF \cite{ietf:groups}.

\paragraph{}
Bei der Elimination existiert bereits eine gültige Spezifikation des Protokolls shim6 durch den RFC 5533 \cite{nordmark:2009:RFC5533},\cite{ietf:documents}. LISP scheint zur Zeit der einzige diskutierte Separationsansatz zu sein. Zu anderen Standards, wie zum Beispiel GSE, Six/One, TRRP oder APT finden sich im Internet-Archiv der IETF keine veröffentlichten Entwürfe. Der Entwurf zu einem LISP-RFC \cite{farinacci:2009:LISP} existiert zur Zeit in der fünften Revision. Er wurde jedoch noch nicht dem Zuständigen Area Director der IETF vorgelegt und auch ein Antrag auf Veröffentlichung existiert noch nicht \cite{ietf:documents}.  Multicast-Erweiterungen für LISP und die Mapping-Systeme befinden sich in einem vergleichbaren oder früheren Stadium. 

\paragraph{}
Eine Diskussion der Lösungsvorschläge findet innerhalb der Routing Reasearch Group (RRG) statt, die Teil der Internet Resarch Task Force ist \cite{irtf:rrg}. Ein mögliches Ergebnis der RRG könnte die Empfehlung einer Lösung an die IETF sein. Letztendlich liegt es jedoch in der Zuständigkeit der AS-Betreiber ein Protokoll auszuwählen und dieses umzusetzen.

\paragraph{}
Sowohl die Separation als auch Elimination lösen die diskutierten Probleme in den Routing-Tabellen der DMZ \cite{jen:2008:start}. Beide beiten Ansatzpunkte, die für Mobile-IP genutzt werden können \cite{farinacci:2009:LISP},\cite{nordmark:2009:RFC5533}. Es gibt aber eine Reihe von Gründen, die für eine Umsetzung des Separationsansatzes sprechen. 

\paragraph{}
Um das Locator-Identifier-Problem zu lösen, speichern beide Ansätze einen Zustand. Beim Eliminationsansatz mit shim6 werden die erreichbaren Adressen der Kommunikationspartner beim Endgerät gespeichert. Diese Speicherung auf der Netzwerkschicht widerspricht dem Prinzip eines zustandsfreien Internets. Der dezentral gespeicherte Zustand erschwert die Suche nach Unregelmäßigkeiten und ist für Betrachter nicht transparent. Der Seperationsansatz speichert die Zuordnung von EIDs und RLOCs zentral im Mapping-System und bietet dem gesamten Internet dieselbe Sicht auf den Status.
Ein weiterer Nachteil der Elimination ist, dass ihre grundlegenden Techniken auf IPv6 basieren. Erst durch IPv6 kann eine Netzwerkschnittstelle per Entwurf mit mehrer Adressen versehen werden. Auch shim6 stellt eine Erweiterung für IPv6 dar. Der Sperationsansatz bietet hier eine generischere Lösung, insbesondere durch die Verwendung von LISP. Aufgrund des Tunneling-Mechanismus müssen keine Annahmen über die Netzwerkebene getroffen werden. Dies unterscheidet LISP auch von Six/One, welches die IPv6-Extension-Header nutzt. IPv6-basierende Lösungen haben vor allem den Nachteil, dass eine flächendeckende Nutzung von IPv6 abgewartet werden muss. 
Der Separationansatz erschafft eine Trennschicht zwischen der DFZ und den Edge-Netzwerken. Dies Trennschicht kann in Zukunft genutzt werden, um Änderungen auf der Netzwerkschicht unabhängig vom jeweils anderen Teil durchführen zu können \cite{jen:2008:start}.
Die Einführung der RLOCs bietet die Möglichkeit die Präfixe in der DFZ neu zu zuordnen, um durch bessere Aggregation eine noch kleinere Routing-Tabelle zu erhalten. 
Die Einführung von Eliminationsprotokollen wie shim6 erfodert die Mitarbeit vom mehr Netzwerkbetreibern als bei den Seperationsprotkollen nötig sind. Werden die Edge-Netzwerkbetreiber gezwungen auf die Vorteile ihrer providerunabhänige Adressen verzichten, könnten sie dies im gewissen Maße als Entmüdigung warhnehmen. Da auch der Haupteil der Kosten bei den Betreibern der Edge-Netzwerke ensteht, sind weitere poltischbedingte Verzögerungen zu erwarten. Die Entwicklungen der letzten Jahre machen aber eine zügige Überarbeitung des Routing-Systems nötig.

\paragraph{}
Es zu erwarten, dass spätestens mit der flächendeckenden Durchsetzung von IPv6 und der damit verbundenen höhren Anzahl an Präfixen sich das Problem weiter verschärfen wird. Dies könnte zusätzlichen Druck auf die Akteure ausüben, und zu einer schnellen Entscheidung führen.

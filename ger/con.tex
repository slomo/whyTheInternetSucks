\section{Bewertung}
Im Rahmen Internet Enigneering Task Force (IETF) werden bereits die einzelden Lösungsansätze, die in dieser Arbeit abgesprochen werden, ausgerarbeiet und diskutiert. Aktuell gibt es sowohl zu LISP als auch zu shim6 aktive Arbeitsgruppen innerhalb der IETF \cite{ietf:groups}. \\

Bei dem Eliminationansatz shim6 existiert bereits der RFC 5533 und damit eine gültige Spezifikation des Protokolls \cite{nordmark:2009:RFC5533},\cite{ietf:documents}. LISP scheint zur Zeit der einzige diskutierte Separationsansatz zu sein. Zu anderen Standarts, wie zum Beispiel GSE, Six/One, TRRP oder APT finden sich im Internet-Archiev der ITEF keine veröffentlichten Entwürfe. Der Entwurf zu einem LISP-RFC \cite{farinacci:2009:LISP} existiert zur Zeit in der 5 Revision. Er wurde jedoch noch nicht dem Zuständigen Area Director der ITEF vorgelegt und auch ein Antrag auf Veröffentlichung existiert nicht \cite{ietf:documents}.  Multicast-Erweiterungen für LISP und die Mappingsysteme befinden sich in einem vergleichbaren oder frühren Stadium. \\

Eine Diskusionen der Lösungsvorschläge findet innerhalb der Routing Reasearch Group (RRG) statt, die Teil der Group Internet Resarch Task Force ist. \cite{irtf:rrg}. Ein mögliches Ergebnis der RRG könnte die Empfehlung einer Lösung an die IETF sein. Letztendlich liegt es jedoch in der Zuständigkeit der AS-Betreiber ein Protokolle auzuwählen und umzusetzen. \\


%TODO: mehr
Obwohl beide Lösungsansätze das prinizpielle Problem der Routing-Tabelle in der DMZ lösen \cite{jen:2008:start}, ist meiner Meinung nach der Sperationansatz überlegen. Zum einen wurde ich überzeugt, dass das verbieten der providerunabhänigen Adressen die Freiheit der Edge-Netzwerkbetreiber einschränkt, und auch eine steigende Komplexität bei den Endgeräten hervor ruft. Zum anderen stellt der Sperationsansatz auch eine Lösung für die Mobiltät von Endgeräten dar und erschafft eine Trennschicht zwischen DMZ und Edge-Netzwerken. Dies Trennschicht kann in Zukunft genutzt werden, um in einem der entstehenden Teile des Internets Protolländerungen unabhänig von der anderen umzusetzen \cite{jen:2008:start}. Zu dem kann die Verteilung der RLOCs genutzt werden um eine bessere Verteilung der Addressen für Provider zu erreichen und somit weniger deaggregirte Prefix entstehen zu lassen.

Der weitere Fortgang der Diskussion bleibt abzuwarten. Es kann aber angenommen werden das mit der flächendeckenden Durchsetzung von IPv6 und der damit verbundenen höhren Anzahl an Präfixen das Problem sich weiter verschärfen wird. Dies könnte zusätlichen Druck auf die Akteure ausüben, und zu einer schellen Entscheidung und Umsetzung führen.




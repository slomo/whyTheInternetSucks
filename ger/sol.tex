\section{Lösungsansätze}
\subsection{Elimination}
Ein vergleichsweise einfacher Lösungsansatz ist die Elimination von Provider unabhängigen Adressen. Dabei werden zukünftig einfach nur noch Adressblöcke von Providern vergeben die in ihrem eigenen Adressblock liegen. Dies führt jedoch zu den anfangs genannten unerwünschten Effekten beim Providerwechsel und der Gewährleitung von Redundanz. Das Problem der Neunummerierung kann relativ elegant gelöst werden durch die Verwendung der Version 6 des Internet Protokolls. Zum einen bietet IPv6 per Design die Möglichkeit mehrere Adresse für ein Gerät zu benutzen, zum anderen beschreibt es die sogenannte "Statless Address Autoconfiguration" mit der es möglich wird ein gesamtes Netzwerk nur die Modifikation an den Routern auf einen weiteren oder anderen Prefix umzustellen \cite{Stockebrand:2006:IPv6}. \\ Um das Problem der Redundanz zu lösen sind schon weiter gehenden Änderungen an diversen Protokollebenen nötig. Zum einem muss eine Maschine innerhalb eines edge networks lernen mit mehreren redundanten Verbindungen umzugehen, zum anderen müssen Mechanismen geschaffen werden über die weitere (alternative) Adressen ausgetauscht werden können, um Maschinen redundant zu erreichen \cite{jen:2008:start}. Denkbar wäre Modifikationen im DNS oder TCP-Protokoll die dies ermöglichen. \\
\paragraph{Vorteile}
Die Lösung ist relativ simple es müssen so gut wie keine neuen Strukturen geschaffen werden, insbesondere keine Mappingdienste wie sie von anderen Lösungsansätzen verwendet werden. 
\paragraph{Nachteile}
Es werden umfangreiche Änderungen an existierende Protokollen nötig, insbesondere muss die Intelligenz der Internetclients steigen, was gerade in Hinblick auf eingebettete Systeme zu Problemen führen kann. Trotz der umfangreichen Änderungen ist die Lösung zur Redundanz nur mäßig. Die nötige Automatisierung der Nummerierung schafft neue Schwachstellen und nimmt den Netzwerkadministratoren die genaue Kontrolle über die verwendeten Adressen. Hinzu kommt das die Routing-Problematik vor allem die ISPs beschäftigt, diese jedoch bei dieser Lösung völlig außen vor gelassen werden. Der meiste Aufwand muss durch die edge networks erbracht werden, ohne das sie einen Vorteil davon erhalten, somit ist nur mit einer sehr langsamen Umstellung zu rechnen. Insgesamt erfordert dieser Vorschlag Änderungen an vielen Komponenten des Internets ohne besonders schöne oder reizvolle Effekte zu erzielen.

\subsection{Seperation}
Der Seperationsansatz wird auch als Identfier/Locator-Split bezeichnet. Einher geht die Überlegung das eine IP-Adresse im klassischen Sinne eine überladene Bedeutung besitzt. Zum einen identifiziert sie ein einzelnes Gerät (oft dauerhaft, mindestens aber immer zu einem definierten Zeitpunkt), gleichzeitig dient sie aber auch da zu ein Gerät zu lokalisieren. Das heißt, verfügt man über eine IP-Adresse kann man sowohl den zugehörigen Rechner aus der Menge aller Rechner wählen (ignoriert man Multicast), gleichzeitig wird aber auch angenommen das die IP-Adresse impliziert wie das Gerät zu erreichen ist, genauer gesagt welche Routing-Entscheidungen auf dem Weg zu ihm zu treffen sind. Dies folgt einer gewissen Analogie zur realen Postanschrift (bestehend aus Name,Strasse,PLZ und Land), birgt allerdings auch die gleichen Nachteile. So ist es nur schwer möglich mehrere Anschriften für ein und das gleiche Gebäude zu haben,eine Anschrift beim Umzug zu behalten oder aber ein Objekt das seinen Standort ändert mit einer Anschrift zu versehen. \\
Die Grundidee der Seperation ist es nun diese zwei Funktionen zu trennen und einen Identfier für ein Gerät und einen Locator für das momentane Netz des Gerätes zu schaffen. Um dies zu erreichen ist es nötig eine Zuordnung zwischen dem Locator und dem Identfiere zu schaffen. Die Art und Weise auf die diese Zuordnung geschaffen wird und die Art der Umwandlung von Identfier in Locator und andersherum unterscheidet die verschiedenen Protokolle die diesen Ansatz verfolgen.

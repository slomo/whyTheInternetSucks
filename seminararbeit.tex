\documentclass[11pt,a4paper]{article}
\usepackage[utf8]{inputenc}
\usepackage[ngerman]{babel}

\author{Yves Müller \\ Freie Universität Berlin \\ Berlin, Deutschland \\ uves@spline.de}
\title{Why the Internet Sucks: A Core Perspective}
\date{}

\begin{document}

\maketitle

\begin{abstract}
Die vorhandene Routing-Struktur des Internets erweist sich zunehmend als unzulänglich. Eines der Hauptprobleme stellt die hohe Belastung der Router in der Default-Free-Zone dar, die unter anderem verursacht wird durch die Nutzung von Multihoming in vielen Teilnetzwerken. Die Lösung dieses und weiterer Probleme erfordert grundsätzliche Änderungen an der vorhandenen Struktur. Dabei bieten sich zwei prinzipielle Lösungsansätze: die Separation und die Elimination. Diese Arbeit soll die Ursachen untersuchen und die Löungsmöglichkeiten vergleichen, um mögliche Perspektiven für die zukünftige Struktus der Internets aufzuzeigen.
\end{abstract}

\tableofcontents

%Einleitungskapitel
\section{Einleitung}

\paragraph{}
Durch die starke Zunahme an Teilnehmern in den letzten Jahren erreicht das Internet seine technischen Grenzen auf der Netzwerkschicht. Probleme zeigen sich unter anderem bei der Anzahl an verfügbaren Adressen, der Mobilität von Endgeräten, der Sicherheit des Gesamtsystems und den Routing-Protokollen. Durch IPv6 wurde für die Anzahl der Adressen und Teilaspekte der Mobilität eine Lösung gefunden. Auch das Thema Sicherheit, als begleitender Prozess jeder Entwicklung, gewann in den letzten Jahren mehr Aufmerksamkeit. Der Handlungsbedarf beim Internet-Routing wurde schon vor einiger Zeit erkannt \cite{deering:1996:map}. Der hohe Verteilungsgrad, die hohe Anzahl an Akteuren und die Einflüsse auf andere Subsysteme des Internets erschwerten die Suche nach neuen Lösungen für das Routing-System. Trotzdem gibt es interessante Lösungsvorschläge, um auch diesen Grundbestandteil des Internets auf zukünftige Anforderungen vorzubereiten. \\
 
Während der letzten Jahre wuchs die Größe der Routing-Tabelle in der Default-Free-Zone (DFZ) exponentiell an, trotz nur linearem Wachstum der verfügbaren Adressen\cite{huston:2001:analyzing}. Da das Routing-System mit wachsender Netzwerkgröße nicht skaliert, ist seine Funktion gefährdet \cite{jen:2008:start}. Einen der größten Faktoren für diesen Anstieg stellt redundante Anbindung von Edge-Netzwerken an das Internet (Multihoming) dar. Neben der steigenden Tabellegröße gefährdet auch die daraus resultierende hohe Frequenz an Updates das Routing-System. Alle bisherigen Lösungsvorschläge fallen in zwei grundsätzliche Kategorien: Separation oder Elimination \cite{jen:2008:start}. \\

Diese Arbeit soll aufzeigen, welche aktuellen Entwicklungen die Skalierbarkeit des Internet-Routing gefährden und die Gründe dafür untersuchen. Es werden Lösungsansätze vorgestellt, die zur Zeit innerhalb von Forschung und Standardisierung diskutiert werden. \\

Der erste Abschnitt soll in das Thema einleiten und die nötigen Begriffe erläutern, um anschließend Probleme sowie deren Ursachen und Auswirkungen genauer zu untersuchen. Zwei prinzipielle Lösungsmöglichkeiten werden im dritten Abschnitt aufgezeigt, wobei Ansätze zur Separation ausführlich diskutiert werden. Schlussendlich werden zuvor gewonnene Erkenntnisse zusammengefasst und bewertet.

 


\section{Problemstellung}
Die beschränkenend Faktoren für das Wachstum des Internet stellen die Anzahl der Addressen dar, sowie die Größe und Anzahl der Updates der Routing Tabelle in der DFZ. Letzteres impliziert vorallem mit wie viel Aufwand die Corerouter ihre eigentlich Arbeit, das Weiterleiten von Paketen durch führen können \cite{Huston:2003:BGP}. Für das Problem der beschränkten Addressierung konnte in den letzten Jahren eine Lösung in der Form von IPv6 gefunden werden, das Problem des Corerouting ist nicht ungelöst jedoch sind die vorhandennen Lösungen umstritten.

\paragraph{Versuch einer Quantfizierung}
Im Jahr 2004 wurden seit den Anfangsjahren des Internets ca. 63000 IP-Blöcke bei reggistriert, davon ca. 18000 in den letzten 7 Jahren \cite{journals/ccr/MengXZHLZ04}. Gleichzeitig enthielt die BGP Routing Tablle über 160000 Einträge, was einer Anzahl von 2,54 Einträgen pro regetrietem Addressblock entspricht, im Veergleich dazu waren es 1998 noch 1,33. Dieser hohe Anzahl an Routen ist vermutlich auf Edge AS zurückzuführen, die durch mehrer ISP angebunden sind \cite{jen:2008:start}. Solche AS werden auch als multihomed bezeichnet. Die Anbindung durch mehere ISPs hat den Vorteil, das sowohl Lastverteilung als auch Redundanz für das eigene Netzwerk gewährleistet werden. \\ 
Neben der hohen Anzahl an Einträgen in den Routingtabellen selbst, bildet die hohe Frequenz von Eintragsänderung ebenfalls ein Problem. Beispielsweise wurden in der ersten Jahreshälfte 2004 24000 Einträge in der BGP-Tabellen entfern, im gleichen Zeitraum wurde jedoch aiuch 36000 Einträge hinzugefügt, die Anzahl der erreichbaren Geräte veränderte sich aber nur geringfügig. Dabei kommt ein hoher Anteil der Änderung von nur wenigen AS, so erreichten vom 14. Dezember bis zum 20. Dezember 2009 insgesamt 465992 BGP updates die BGP Tabellen, wovon 30,33 \% von nur 50 AS getätigt wurden \cite{Huston:aktuell:BGP}. Dies könnte ein Hinweis auf gößere Umzüg zumeist kleiner ASs oder auf fehlkonfigurierte BGP-Router sein. \\
Es ist fraglich für wieviele der betroffenden Router die überhaupt eine der prpagierten Routen wichtig ist, in dem Sinne das sie auch tatsächlich benutzt wird.



\section{Lösungsansätze}
\subsection{Elimination}
Ein Lösungsansatz ist die Elimination von Provider unabhängigen Adressen. Dabei vergeben Provoider nur noch Adressblöcke, die in ihrem eigenen Adressblock liegen. Diese hierarchische Vergabe verlangt neue Ansätze bei Multihoming und Providerwechsel. Diese Probleme können gelöst werden durch die Verwendung der Version 6 des Internet Protokolls. IPv6 bietet per Entwurf die Möglichkeit mehrere Adresse für eine Netzwerkschnittstelle zu benutzen und es beschreibt die sogenannte "Statless Address Autoconfiguration", mit der es möglich wird ein gesamtes Netzwerk auf einen weiteren oder anderen Prefix umzustellen, ohne Änderung an jedem Netzwerkgerät vorzunehmen. \cite{RFC4862}. \\ Um das Problem der Redundanz zu lösen sind  weitergehenden Änderungen an diversen Protokollebenen nötig. Zum einem muss eine Maschine innerhalb eines Edge-Networks lernen mit mehreren redundanten Verbindungen umzugehen, zum anderen müssen Mechanismen geschaffen werden über die weitere (alternative) Adressen ausgetauscht werden können, um Maschinen redundant zu erreichen \cite{jen:2008:start}. Denkbar wäre Modifikationen im DNS oder Transportlayer-Protokoll die dies ermöglichen.
\paragraph{Vorteile}
Dieser Lösungsansatz kann durch Modifikation an vorhandenen Strukturen umgesetzt werden. Es werden keine neuen Protokolle oder Instanzen benötigt.
\paragraph{Nachteile} %TODO: rewrite?
Es werden umfangreiche Änderungen an existierende Protokollen nötig, insbesondere die Intelligenz der Endgeräte muss erhöht, was gerade in Hinblick auf eingebettete Systeme zu Problemen führen kann. Die nötige Automatisierung der Nummerierung schafft neue Schwachstellen und nimmt den Netzwerkadministratoren die genaue Kontrolle über die verwendeten Adressen. Hinzu kommt das die Routing-Problematik vor allem die ISPs beschäftigt, diese jedoch bei dieser Lösung völlig außen vor gelassen werden. Der meiste Aufwand muss durch die edge networks erbracht werden, ohne das sie einen Vorteil davon erhalten, somit ist nur mit einer sehr langsamen Umstellung zu rechnen. % TODO: Fazit

\subsection{Seperation}
Der Seperationsansatz wird auch als Identifier/Locator-Split bezeichnet. Eine Unicast-Adresse besitzt in der Praxis eine überladene Bedeutung. Zum einen identifiziert sie ein einzelnes Interface zu einem bestimmten Zeitpunkt, gleichzeitig dient sie aber auch dazu ein Gerät zu lokalisieren. Da eine Verbindung zwischen zwei Geräten an die Addressen der jeweiligen Gesprächspartner gebunden ist, bedeutet ein Wechsel des Netzes auch den Abbruch der Verbindung. Dies schränkt ins besondere Anwendung wie Mobile-IP ein. \\
%TODO: Frage: was heißt das für Multihoming und Providerwechsel.

Die Grundidee der Seperation ist es diese zwei Funktionen zu trennen und einen Identfier für ein Gerät und einen Locator für den Ort des Gerätes im Internet zu schaffen. Routing-Entscheidungen im Internet-Core werden anhand von RLOC (routing Locators) getroffen. Ein Endgerät in einem Edge-Network wir durch EID (Endpoint Identfier) eindeutig und dauerhaft identifziert. An der Schnittstelle zwischen EN und ISP müssen EIDs in RLOCs und anderesherum übersetzt werden. Um dies zu ermöglichen muss eine Zuordnung zwischen dem Locator und dem Identfierer geschaffen werden. Dazu wird ein Mappingservice verwendet, der Auskunft über welche RLOCs ein gegebner EID zuerreichen ist. 

\paragraph{Vorteile}
Innerhalb des Internet-Cores müssen nur ISPs mit RLOCs addressiert werden. Dies führt dazu das die Routing-Tabelle im Core erheblich kleiner wird und stabiler, da es nur weniger ISPs als Edge-Networks gibt und diese auch nur geringen Änderungen unterliegen. \\
Um ein Edge-Network durch mehrer Provider anzubinden, also um Multihoming zu betreiben, müssen nur die RLOCs der Provider im Mappingsystem mit EIDs des Edge-Networks assoziert werden. Dies ermöglich Multihoming ohne Auswirkungen auf des Routing im Internet-Core zu haben. Auch der Providerwechsel wir vereinfacht, da dieser nur einer Änderung der Zuordnung im Mappingsystem entspricht. \\
Durch die den Identfier/Locator-Split wird es möglich das sich Endgeräte bewegen ohne das sich ihr Identfier ändert, so dass Verbindungen auffrecht erhalten können. Dies bedingt jedoch ein ausreichend schnelles Mappingsystem, da ohne korrekte Zuordnung keine Daten zum Endgerät geleitet werden können. \\
Da EIDs und RLOCs sich ansonsten ähnlich zu jetzigen IP-Addressen verhalten ist es weder notwenidig den Endgeräte zu modifzieren, noch die Router innerhalb des Internet-Core zu verändern. Nur an der Übergabe zwischen Internet-Core und Edge-Network sind Änderungen nötig. 
% TODO: More benefits from startpaper..

\paragraph{Nachteile}
Um die beschriebenen Eigenschaften zu erhalten muss ein Mappingdienst geschaffen werden, der einer Reihe Anfoderungen entspricht: Er muss schnell die gesuchte Zuordnung liefern, um Paketverlust zu vermeiden. Updates des Mappings müssen sofort im gesamten Dienst verbreitet werden, um Endgeräte erreichen zu können. Dezentralität und Robustheit werden benötigt, da ein Ausfall des Mappingdienst den Ausfall des Internets bedeuten würde. Die Korrektheit von Zuordnungen muss gewährleistet sein um den korrekten Empfänger zu erreichen. Es werden also Anwendungen an den Mappingdinest gestellt die von keinem existierenden Dienst in aussreichendem Maße gewährlseistet werden.

\paragraph{} % TODO: add proper source 
Um die Umwandlung zwischen RLOCs und EDIs zu realsieren kann eine IP-in-IP Kapslung verwendet werden, man spricht in diesem Fall von Map-and-Encap. Eine Standart der dies beschreibt ist LISP. 

\subsubsection{Map\&Encap am Beispiel von LISP}
LISP erfodert keine Modifikationen an den Endgeräten, in den ENs. Sie arbeiten bei der Addresierung weiterhin mit IPv4 oder IPv6 Addressen. Auch die Namensauflösung und das Routing innerhalb von Edge-Networks muss nicht verändert werden. Wir ein Paket zum ISP übermittelt erreicht es beim Provider den ITR (Ingress Tunnel Router). Dieser ermitelt zum Ziel-EID die RLOCs mittels des Mappingsystems. Anschlißend packt er das IP-Paket in ein LISP-Paket. Das LISP-Paket ist ein IP/UDP-Packet mit zusätlichen Lisp-Header. Als Zieladdress wird die zuvor ermittelte RLOC benutzt, Quelladdresse ist die RLOC des ITRs. Anschlißend wird das Packet durch den Internet-Core übermittelt. Es erreicht schließlich einen ETR (Egress Tunnel Router) der die ermittelte RLOC besitzt. Dieser ist an das EN in dem sich die Ziel-EID befindet angeschlossen. Er packt das Packet aus und übermittel es an das EN. \\
\paragraph{Fragmentation} % TODO: Sprach verbessern.
Durch die Kapselung des Ursprungspaketes kann es das zu kommen, dass das enstehenden Paket die MTU (Maximum Transfer Unit) einer Verbindung innerhalb des Internetcores überschreitet. LISP definiert zwei verscheidene Verhfahren um dieses Problem zu lösen. Um das Problem statusfrei und damit möglichst einfach zu behandeln wird empfohlen Pakete die eingepackt eine gewisse MTU übersteigen zu verwerfenen und die entsprechende ICMP Meldung zum Sender zu schicken. Die zweite Lösung basiert darauf das ein ITR in der Praxis einige Zuordnungen vorhalten wird um nicht immer wieder das Mapping zu erfragen. Zu diesen bekannten Mapping wird dann die MTU gespeichert, bei der eine ICMP-Too-Big Nachricht empfangen wurde. Pakete zu diesem RLOC die diese MTU übersteigen werden ebenfalls mit der ICMP-Too-Big Nachricht beantwortet.
% TODO: Mehr LISP Beschreibung wenn nötig.

\subsubsection{Address-Rewriting am Beispiel von Six/One Router}
Im Gegensatz zu LISP wir bei Six/One das umwandeln zwischen EIDs und RLOCs von Routern innerhalb des Edge-Networks übernommen. Desweitern wird darauf verzichtet die Pakete zu kapseln, sondern direkt die Paket Addresse verändert. 
 
\subsubsection{Mappingsystem}




\section{Zusammenfassung und Bewertung}
\paragraph{}
Im Rahmen der Internet Enigneering Task Force (IETF) wurden bereits die einzelden Lösungsansätze, die in dieser Arbeit angesprochen werden, ausgearbeitet und diskutiert. Aktuell gibt es sowohl zu LISP als auch zu shim6 aktive Arbeitsgruppen innerhalb der IETF \cite{ietf:groups}.

\paragraph{}
Bei der Elimination existiert bereits eine gültige Spezifikation des Protokolls shim6 durch den RFC 5533 \cite{nordmark:2009:RFC5533},\cite{ietf:documents}. LISP scheint zur Zeit der einzige diskutierte Separationsansatz zu sein. Zu anderen Standards, wie zum Beispiel GSE, Six/One, TRRP oder APT finden sich im Internet-Archiv der IETF keine veröffentlichten Entwürfe. Der Entwurf zu einem LISP-RFC \cite{farinacci:2009:LISP} existiert zur Zeit in der fünften Revision. Er wurde jedoch noch nicht dem Zuständigen Area Director der IETF vorgelegt und auch ein Antrag auf Veröffentlichung existiert noch nicht \cite{ietf:documents}.  Multicast-Erweiterungen für LISP und die Mapping-Systeme befinden sich in einem vergleichbaren oder früheren Stadium. 

\paragraph{}
Eine Diskussion der Lösungsvorschläge findet innerhalb der Routing Reasearch Group (RRG) statt, die Teil der Internet Resarch Task Force ist \cite{irtf:rrg}. Ein mögliches Ergebnis der RRG könnte die Empfehlung einer Lösung an die IETF sein. Letztendlich liegt es jedoch in der Zuständigkeit der AS-Betreiber ein Protokoll auszuwählen und dieses umzusetzen.

\paragraph{}
Sowohl die Separation als auch Elimination lösen die diskutierten Probleme in den Routing-Tabellen der DMZ \cite{jen:2008:start}. Beide beiten Ansatzpunkte, die für Mobile-IP genutzt werden können \cite{farinacci:2009:LISP},\cite{nordmark:2009:RFC5533}. Es gibt aber eine Reihe von Gründen, die für eine Umsetzung des Separationsansatzes sprechen. 

\paragraph{}
Um das Locator-Identifier-Problem zu lösen, speichern beide Ansätze ei\-nen Zustand. Beim Eliminationsansatz mit shim6 werden die erreichbaren Adressen der Kommunikationspartner beim Endgerät gespeichert. Diese Speicherung auf der Netzwerkschicht widerspricht dem Prinzip eines zustandsfreien Internets. Der dezentral gespeicherte Zustand erschwert die Suche nach Unregelmäßigkeiten und ist für Betrachter nicht transparent. Der Seperationsansatz speichert die Zuordnung von EIDs und RLOCs zentral im Mapping-System und bietet dem gesamten Internet dieselbe Sicht auf den Status.
Ein weiterer Nachteil der Elimination ist, dass ihre grundlegenden Techniken auf IPv6 basieren. Erst durch IPv6 kann eine Netzwerkschnittstelle per Entwurf mit mehrer Adressen versehen werden. Auch shim6 stellt eine Erweiterung für IPv6 dar. Der Sperationsansatz bietet hier eine generischere Lösung, insbesondere durch die Verwendung von LISP. Aufgrund des Tunneling-Mechanismus müssen keine Annahmen über die Netzwerkebene getroffen werden. Dies unterscheidet LISP auch von Six/One, welches die IPv6-Extension-Header nutzt. IPv6-basierende Lösungen haben vor allem den Nachteil, dass eine flächendeckende Nutzung von IPv6 abgewartet werden muss. 
Der Separationansatz erschafft eine Trennschicht zwischen der DFZ und den Edge-Netzwerken. Dies Trennschicht kann in Zukunft genutzt werden, um Änderungen auf der Netzwerkschicht unabhängig vom jeweils anderen Teil durchführen zu können \cite{jen:2008:start}.
Die Einführung der RLOCs bietet die Möglichkeit die Präfixe in der DFZ neu zu zuordnen, um durch bessere Aggregation eine noch kleinere Routing-Tabelle zu erhalten. 
Die Einführung von Eliminationsprotokollen wie shim6 erfodert die Mitarbeit vom mehr Netzwerkbetreibern als bei den Seperationsprotkollen nötig sind. Werden die Edge-Netzwerkbetreiber gezwungen auf die Vorteile ihrer providerunabhänige Adressen verzichten, könnten sie dies im gewissen Maße als Entmüdigung warhnehmen. Da auch der Haupteil der Kosten bei den Betreibern der Edge-Netzwerke ensteht, sind weitere poltischbedingte Verzögerungen zu erwarten. Die Entwicklungen der letzten Jahre machen aber eine zügige Überarbeitung des Routing-Systems nötig.

\paragraph{}
Es zu erwarten, dass spätestens mit der flächendeckenden Durchsetzung von IPv6 und der damit verbundenen höhren Anzahl an Präfixen sich das Problem weiter verschärfen wird. Dies könnte zusätzlichen Druck auf die Akteure ausüben, und zu einer schnellen Entscheidung führen.


\bibliographystyle{plain}
\bibliography{tau}

\end{document}

